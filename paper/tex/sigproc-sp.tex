% THIS IS SIGPROC-SP.TEX - VERSION 3.1
% WORKS WITH V3.2SP OF ACM_PROC_ARTICLE-SP.CLS
%APRIL 2009

\documentclass{acm_proc_article-sp}

\begin{document}

\title{NV: Nessus Vulnerability Visualization for the Web}

\numberofauthors{5} 

\author{
%\alignauthor
%Lane Harrison
%       \affaddr{Oak Ridge National Lab}\\
%       \email{trovato@corporation.com}
% 2nd.
author
}

\maketitle
\begin{abstract}
\end{abstract}

% A category with the (minimum) three required fields
%\category{H.4}{Information Systems Applications}{Miscellaneous}
%A category including the fourth, optional field follows...
%\category{D.2.8}{Software Engineering}{Metrics}[complexity measures, performance measures]

%\terms{Theory}

%\keywords{ACM proceedings, \LaTeX, text tagging} % NOT required for Proceedings

\section{Introduction}

\section{Related Work}
Currently most computer vulnerability analysis is done using graph based
techniques to model the state of the system.
 One such technique is known as
Topological Vulnerability Analysis (TVA).
 TVA uses the network state and attack
vectors between machines to create an attach graph that will model all possible attack paths in a
network.
 To generate these attack graphs TVA uses information from scanning
tools such as Nessus and Retina.
 The graphs generated by TVA tend to be very
large so it introduces an aggregation and visual analysis element to make the
models easier to comprehend by an analyst.
 One aggregation used by the TVA
visualization is to
aggregate machines based on their ability to access other machines.
 A group of
machines will be aggregated if each node in the group has access to every other
node.
These groupings are then aggregated into a single node in the visualization.
**Noel**

Researchers have also used model checking tools like NuSMV to manipulate graph
representations of a network where each node is a state of the network and each
transition represents an exploit.
 This type of attack graph allows and analyst
to focus efforts on patching exploits (edges) that create the largest
disconnects in the graph.
 This type of analysis is convenient because we
already have graph algorithms that can efficiently perform such analysis.
**Aman**

Ou, Govindavjhala and Appel take a different approach to security analysis in
their MulVAL project.
They attempt to model the interactions between known vulnerabilities and
software bugs, configurations and permission policies.
 In their approach an
analyst will specify the system and policies in a logic language that is a subset of the
Prolog logic programming language and vulnerabilities in the Open Vulnerability
Assessment Language.
After the systems, policies and vulnerabilities are defined
the MulVAL system uses a two phase algorithm to simulate attacks and then policy
checking.
 The system generates all possible attacks based on the
vulnerabilities and then compares those with the defined policies to detect
violations.
**Ou**

\section{System Design}
The goal of nv is to support the sysadmin's understanding of vulnerabilities in their network by combining the results of a Nessus scan in raw format and (optionally) a list of critial machines in their network into an interactive visualization.
This visualization is designed to support common workflows in vulnerability discovery, analysis, and mitigation.
Some of these are described in the Case Studies sections.
This section covers the visualization and interaction design.

\subsection{Data}
Nessus data in detail

The Nessus scan results provide significant detail about the state of all machines on the specified network.
(TODO talk about how the scan actually works?) This information includes the port the vulnerable service is running on, what service and what version is running, what other versions of this software share this vulnerability, and a general description of the vulnerability.
These results also indicate whether this is an actual vulnerability or just a general security notice, and it also provides a severity score and several unique identifiers related to this vulnerability, which can be used to find additional information.
The results also often give information about how this vulnerability can be patched or otherwise mitigated.
Figure nnnnnnnnnnnnnnnnnnnnnnnnnnnnn (TODO) shows an example; this particular example is from the VAST Challenge 2011 data set.

results|192.168.2|192.168.2.175|cifs (445/tcp)|46844|Security Hole|Synopsis :\n\nThe remote Windows host contains a font driver that is affected by\na privilege escalation vulnerability.\n\nDescription :\n\nThe remote Windows host contains a version of the OpenType Compact\nFont Format (CFF) Font Driver that fails to properly validate certain\ndata passed from user mode to kernel mode.\n\nBy viewing content rendered in a specially crafted CFF font, a local\nattacker may be able to exploit this vulnerability to execute \narbitrary code in kernel mode and take complete control of the \naffected system.\n\nSolution :\n\nMicrosoft has released a set of patches for Windows 2000, XP, 2003,\nVista, 2008, 7, and 2008 R2 :\n\nhttp://www.microsoft.com/technet/security/Bulletin/MS10-037.mspx\n\nRisk factor :\n\nHigh / CVSS Base Score : 9.3\n(CVSS2#AV:N/AC:M/Au:N/C:C/I:C/A:C)\n\n\nPlugin output :\n- C:\\WINDOWS\\System32\\Atmfd.dll has not been patched\n    Remote version : 5.1.2.226\n    Should be : 5.1.2.228\n\n\n\nCVE : CVE-2010-0819\nBID : 40572\nOther references : OSVDB:65217,MSFT:MS10-037\n

\subsection{Use Case}
The primary goal of nv is to support sysadmins in identifying and analyzing vulnerabilities in their network, information which they may then use to better prioritize their (often) limited resources.
Specifically, the main questions nv seeks to answer are as follows:

- What vulnerabilities are most common across the network?
- What machines or groups have the most severe vulnerabilities?
- Are critical machines are vulnerable?

Next, we describe the visualizations and interactions necessary to support these tasks.

\subsection{Visualization and Interaction}
Nv consists of multiple coordinated views including a treemap, several histograms, and a detail-information area showing information on the selected Nessus id.
Each of these are designed to support a specific aspect of the vulnerability analysis workflow

Our primary visualization is a zoomable treemap (TODO cite).
We chose to use a treemap over other hierarchical visualization methods such as network/tree-layouts for several reasons.
First, our goal with nv is to support the analysis of Nessus scans on large networks.
While information on the network topology is useful for vulnerability analysis, it is important to note that in large dynamic networks, a complete network topology is often either unavailable or too large to be visualized directly.
The space-filling aspects of treemaps make them more scalable in this regard.
Another reason we used treemaps was for their ability to effectively make use of both size and color for encoding data attributes.

Since Nessus data is not stored in a hierarchical form by default, it could be visualized using many multi-dimensional visualization techniques, such as parallel coordinates or scatterplot matrices.
However, because the scalability of the visualization was a primary concern, we opted to nest the data from individual vulnerabilities and ports up to IPs and groups of IPs.


We also use data-accumulation and coloring methods to ensure that data is not obscured by the hierarchy.
For instance, when comparing two Nessus scans, nodes are colored by the maximum count of issue states (fixed, open, or new issues) in their child nodes.
A potential disadvantage of this approach is that a node could contain sligthly more fixed issues than open issues, and yet will still be colored green.
To alleviate this problem, we add the option to split the nodes by issue-state higher in the hierarchy.
Both options are shown in figure (TODO make figure).



**Screen shot of state_issue here**
![Alt Text](screenshots/state_issue.png)

The advantage to separating issue-states higher is that the analyst can explore only the fixed issues or only the open issues.
However, the disadvantage of this approach is that the IPs are then separated since they can appear in any branch of the hierarchy (fixed, open, and new).
To our knowledge, there exists no widely accepted visual technique that can effectively represent multiple attributes at every level in a treemap.
However, we plan to explore other common approaches such as glyphs and combined color scales in future versions of nv.


Since analysts can specify the criticality of both individual machines and groups of machines in nv, the treemap includes sizing by criticality as an option.
The most critical machines therefore appear as larger nodes, while still being colored by severity.
Other sizing options include severity (the default) and by issue counts.
Dual encoding severity with both color and size can be useful, as the darkest colored and largest nodes appear at the top left in each level of the histogram.

The color scales in the treemap were created using ColorBrewer2 (TODO cite).
While the primary color scales shown in the paper are designed to have semantic meanings (green for fixed, red for new, orange for open), we also include a colorblind-safe version, which is shown in figure (TODO figure).


**Screen shot of cb here**
![Alt Text](screenshots/cb_version_both.png)

Nv includes several histograms, including issue-type (note, hole, or open port), severity (CVSS score), top Nessus note ids, and top Nessus hole ids.
These histograms serve dual purposes, as both overviews of the data and as filters by which sysadmins may guide their analysis.
For instance, by brushing over the highest values in the severity histogram, the appropriate nodes in the treemap are highlighted.
This works by examining each child of each element in the current level of the hierarchy.
Another use of the histograms is to easily highlight the most commonly occuring issues in the network.
A possible drawback of this approach is that sometimes the least common issues can be the most damaging.
However, this issue is mitigated by the fact that the treemap can be be sized and colored by severity, which makes the most damaging issues easy to find.
The histograms also operate in as conjunction (AND), meaning that the sysadmin can specify queries such as all issues of type hole with severity of 5 or greater.

The Nessus information area is updated when the sysadmin drills down to the level at which Nessus issue-identification numbers are shown.
The area then updates with detailed information about the currently selected Nessus id, including a synopsis, detailed description, vulnerability family, and solution (when available).
Based on this information, the sysadmin has the option to mark the vulnerability as either fixed or as a non-issue, which re-colors the node in the treemap.
This functionality is intended to serve as a way for analysts to avoid revisiting issues that have been addressed.


\subsection{Implementation}

One significant requirement for this project was to not unnecessarily disclose the Nessus scan results to any third parties; because this information would be very valuable to any attacker, the users of this tool would have an obvious concern to prevent its disclosure.
To address this concern, the NV tool runs entirely in the browser client, without relying on any server-side functionality, and without loading any non-local resources.
We were able to achieve this in a highly scalable implementation by combining several existing components, including the crossfilter data model library and the d3 library for data-driven DOM manipulation.
We also developed a custom parser for the .nbe files, and related code to compare and merge these results.
 We were also able to handle these tasks in the browser with good performance.
For additional peace-of-mind to any users, the entire technology stack is open source, and the NV tool itself will also soon be open sourced.(TODO make less 'meh')

One difficulty caused by the requirement of not leaking scan results was how to look up additional details about the results.
Nessus provides an interface to access significant additional information about any specific vulnerability ID, including useful details such as related CVE and Bugrtaq IDs, and information about how to patch or otherwise address each issue.
However, using this directly could still give an adversary significant information; if they could observe any of this traffic, then they could still learn which vulnerabilities are present.
To address this, we build a local cache of this information, which the client can access offline.
(TODO presumably we won't be open-sourcing this part, heh.
I guess that's obvious enough that we don't need to say it...)

The main treemap and the histograms were created using the d3 library (TODO cite), which is decigned for "apply[ing] data-driven transformations" to the Document Object Model (DOM).
 D3 is fast, flexible, and supports large datasets, which were our main requirements.
(TODO elaborate?  Should we maybe say that it's awesome but also a pain?  Not sure where to go with this ...)

The crossfilter library (TODO cite), designed for accessing "large multivariate datasets in the browser", was used to store and access our Nessus scan results and all related information about the machines and subnets on the network.
 This handles the data entirely in memory, and handles storage and access in an efficient manner.
 (TODO same issue as above.)

As with everything, jquery was used for massive convenience when manipulating elements and such.
(TODO what else can we actually say about it?  TODO merge these 3 into one paragraph?)


\section{Case Study 1: Dynamic Vulnerability State Network }

\section{Case Study 2: Static Vulnerability State Network }

\section{Conclusion and Future Work}

We have introduced nv, a Nessus vulnerability visualization system.

Nv is designed to support sysadmins in the tasks of vulnerability discovery, analysis, and management through an interactive visualization.


%\end{document}  % This is where a 'short' article might terminate

\bibliographystyle{abbrv}
\bibliography{sigproc}  % sigproc.bib is the name of the Bibliography in this case

\subsection{References}
\end{document}

%%% Sample citation, table, conclusion
%\subsection{Citations}
%Citations to articles \cite{bowman:reasoning, clark:pct, braams:babel, herlihy:methodology},
%conference
%proceedings \cite{clark:pct} or books \cite{salas:calculus, Lamport:LaTeX} listed
%
%\begin{table*}
%\centering
%\caption{Some Typical Commands}
%\begin{tabular}{|c|c|l|} \hline
%Command&A Number&Comments\\ \hline
%\texttt{{\char'134}alignauthor} & 100& Author alignment\\ \hline
%\texttt{{\char'134}numberofauthors}& 200& Author enumeration\\ \hline
%\texttt{{\char'134}table}& 300 & For tables\\ \hline
%\texttt{{\char'134}table*}& 400& For wider tables\\ \hline\end{tabular}
%\end{table*}
%% end the environment with {table*}, NOTE not {table}!
%
%\begin{figure}
%\centering
%\epsfig{file=fly.eps}
%\caption{A sample black and white graphic (.eps format).}
%\end{figure}


